\documentclass[letterpaper, 12pt]{article}
\usepackage{amsmath, amssymb, mathtools} % essentials
\usepackage[left=1in,right=1in,top=1in,bottom=1.5in]{geometry}
\renewcommand{\baselinestretch}{1.1}
\usepackage{enumerate}
\usepackage{float}
\usepackage{amsthm}
\newtheorem{thm}{Theorem}
\newtheorem{conj}{Conjecture}
%\usepackage{tikz-cd}
%\usepackage{xcolor}
%\usepackage{url}
%\usepackage{graphicx}
%\graphicspath{{./}}
%\usepackage{enumitem}
%\usepackage{comment}

% convenience names
\newcommand{\nequiv}{\not\equiv}
\newcommand{\nsubset}{\not\subset}
\renewcommand{\emptyset}{\varnothing}
\newcommand{\xto}{\xrightarrow}
\newcommand{\injto}{\hookrightarrow}
\newcommand{\surjto}{\twoheadrightarrow}

\newcommand{\diff}{\mathop{}\!d} % as in $\int e^x \diff x$

\renewcommand{\Re}{\operatorname{Re}}
\renewcommand{\Im}{\operatorname{Im}}
\DeclareMathOperator{\Cl}{Cl} % ideal class group
\DeclareMathOperator{\Frac}{Frac}
\DeclareMathOperator{\Tr}{Tr}
\DeclareMathOperator{\Gal}{Gal}
\DeclareMathOperator{\Stab}{Stab}

% code from https://tex.stackexchange.com/a/2610/223981 that fixes a spacing issue with \left\right
\let\originalleft\left
\let\originalright\right
\renewcommand{\left}{\mathopen{}\mathclose\bgroup\originalleft}
\renewcommand{\right}{\aftergroup\egroup\originalright}

\title{15-Theorem and Its Generalizations: A Journey from Ramanujan to Conway and Bhargava\footnote{Fall 2024 Math 254A Term Paper. I wrote this myself, with help from Google Gemini AI for generating citations.}: Presentation Edition}
\author{Mingyang Cen \\ Student ID: 3036462810 \\ Email address: \texttt{mcen@berkeley.edu}}

\begin{document}

\maketitle

\section{Introduction}
Lagrange's famous four-square theorem states that the quadratic form $x^2+y^2+z^2+w^2$ represents all natural numbers. One may wonder which other quadratic forms have this property of representing all natural numbers. A solution is provided by the 15-Theorem and 290-Theorem, which I will discuss in this talk.

First, we need to specify which quadratic forms we are looking at. For a quadratic form, ``integer-matrix'' means that when the quadratic form is represented as a symmetric matrix, all entries are integers. ``Integer-valued'' is a more lenient condition that allows off-diagonal entries to be half-integers.

The 15-Theorem states that a positive-definite integer-matrix quadratic form (which I will refer to as simply a quadratic form from now on) represents all positive integers exactly when it represents 1, 2, 3, 5, 6, 7, 10, 14, and 15. These nine numbers are called critical numbers.

I have created SageMath code for some computations required for the 15-theorem.

\section{Definitions}
There is a well-known natural bijection between lattices with integer inner products and positive-definite integer-matrix quadratic forms, so any operation on a lattice can be applied to a quadratic form, and vice versa. I am going to use lattices and quadratic forms in the proof, whichever is most convenient.

I will present several important pieces of machinery for the proof. A quadratic form or its corresponding lattice is defined to be \emph{universal} if it represents all positive integers. If it is not universal, its \emph{truant} is defined to be the smallest positive integer not represented. An \emph{escalation} of a lattice $L$ is a lattice generated by $L$ and a vector of norm equal to the truant of $L$. An \emph{escalator lattice} is one reached from a series of escalations from the zero lattice. We will consider escalator lattices only, because they contain the necessary information for all quadratic forms.

\section{Proof of the 15-Theorem}
\subsection{Escalators of dimension 1, 2, and 3}
Through computation, there are 9 three-dimensional escalators.
Escalating again gives 207 nonisomorphic four-dimensional escalator lattices. An important issue is to determine that the quaternary escalators either have truant in the critical numbers or are universal.

\subsection{Escalators of dimension 4}
We want to show 201 of the 207 quaternary escalators are universal, and the other 6 have truants 10 or 15.

As an illustrative example for the argument for the 201 universal quaternaries, let $L$ be an escalation of $L' = \begin{bsmallmatrix} 1 & 0 & 0 \\ 0 & 2 & 0 \\ 0 & 0 & 2 \end{bsmallmatrix}$, where the orthogonal complement of $L'$ in $L$ has Gram matrix with one entry $m$. Since $L'$ is unique in its genus, we may apply the Hasse-Minkowski principle to go from the global setting of the integers to the local setting of the $p$-adic numbers. Through some application of Hensel's lemma, we find that $L'$ represents everything but positive integers of the form $4^a (8b + 7)$, where $a$ and $b$ are nonnegative integers.

Suppose for the sake of contradiction that $L$ is not universal, and let its truant be $u$. Since $L'$ cannot represent $u$, we can write $u$ in the form $4^a (8b + 7)$. Since $u$ is minimal, it is squarefree, so $a = 0$ and $u \equiv 7 \pmod 8$. We consider three cases depending on $m$ modulo $8$:
\begin{itemize}
    \item If $m$ is not congruent to $0$, $3$, or $7$ modulo $8$, then $u-m$ cannot be of the form $4^a (8b + 7)$.
    \item If $m \equiv 3, 7 \pmod 8$, then $u-4m$ cannot be of the form $4^a (8b + 7)$ either.
    \item If $m \equiv 0 \pmod 8$, then $L$ is called exceptional, and we have to use a different $L'$.
\end{itemize}

Therefore, either $u < 4m$ or $u - m$ or $u - 4m$ (depending on $m \pmod 8$) is represented by $L'$. Because $L = L' \oplus [m]$, if $u - m$ or $u - 4m$ is represented by $L'$, $u$ is represented by $L$. This is false by assumption. Hence, $u$ must be less than $4m$. By computation, $m \le 28$, so $u < 112 = 4 \cdot 28$. Now that we have bounded $u$, we use computer checks to show that $L$ represents all positive integers up to $112$. This shows $u$ did not exist in the first place, so $L$ is universal.

There are a few special cases where a slightly modified argument must be applied, but in the end, 201 quaternary escalators are shown to be universal.

\subsection{Escalators of dimension 5}
The $6$ nonuniversal four-dimensional escalators have truants $10,10,15,15,15,15$. It is impractical to consider every single one of the 1630 five-dimensional escalators, but we have a trick. We may apply the same argument as we just did for the other 201 four-dimensional escalators, but the final check that $L$ represents all positive integers up to some number fails with just one number. This shows that these $6$ lattices each miss that number and no others. That number is the lattice's truant, so the $5$-dimensional escalators of the lattice must be universal. This finishes our analysis of escalators.

For every quadratic form $Q$ representing each critical number, there exists an escalator within $Q$ representing every critical number. Through analysis of the escalators, we find that the escalator within $Q$ is in fact universal. Hence, $Q$ is universal, proving the 15-theorem.

The number 15 is special in this theorem since, in fact,
a quadratic form representing all critical numbers besides $15$ must represent all positive integers besides $15$.
Also, by application of the 15-Theorem, we easily obtain an extremely useful classification of universal four-dimensional quadratic forms, namely, that there are exactly $204$ non-isomorphic universal four-dimensional quadratic forms.

The 15-Theorem also provides a shortcut to classifying all universal $4$-variable positive-definite integer-matrix quadratic forms. We find a bound on the discriminant of such a quadratic form and systematically apply the $15$-Theorem in each case to find there are exactly $204$ non-isomorphic universal quaternaries.

\section{Related results}
The 290-Theorem states the same thing as the 15-Theorem, except we allow integer-valued quadratic forms, and the critical numbers are enlarged to a set of 29 numbers, up to 290.
The 290-Theorem appears to be similar to the 15-Theorem, and the proof has the same structure. However, the weakened condition of being integer-valued vastly increases the number of cases that must be considered. Because of this, the 290-Theorem requires several analytic results on modular forms for the proof.
There is a similar classification of universal four-dimensional integer-valued quadratic forms.

Several generalizations of the 15-Theorem and the 290-Theorem exist: to quadratic forms representing odd numbers, prime numbers, numbers coprime to a certain number, etc.

\end{document}
