\documentclass[letterpaper, 12pt]{article}
\usepackage{amsmath, amssymb, mathtools} % essentials
\usepackage[left=1in,right=1in,top=1in,bottom=1.5in]{geometry}
\renewcommand{\baselinestretch}{1.1}
\usepackage{enumerate}
\usepackage{float}
\usepackage{amsthm}
\newtheorem{thm}{Theorem}
\newtheorem{conj}{Conjecture}
%\usepackage{tikz-cd}
%\usepackage{xcolor}
%\usepackage{url}
%\usepackage{graphicx}
%\graphicspath{{./}}
%\usepackage{enumitem}
%\usepackage{comment}

% convenience names
\newcommand{\nequiv}{\not\equiv}
\newcommand{\nsubset}{\not\subset}
\renewcommand{\emptyset}{\varnothing}
\newcommand{\xto}{\xrightarrow}
\newcommand{\injto}{\hookrightarrow}
\newcommand{\surjto}{\twoheadrightarrow}

\newcommand{\diff}{\mathop{}\!d} % as in $\int e^x \diff x$

\renewcommand{\Re}{\operatorname{Re}}
\renewcommand{\Im}{\operatorname{Im}}
\DeclareMathOperator{\Cl}{Cl} % ideal class group
\DeclareMathOperator{\Frac}{Frac}
\DeclareMathOperator{\Tr}{Tr}
\DeclareMathOperator{\Gal}{Gal}
\DeclareMathOperator{\Stab}{Stab}

% code from https://tex.stackexchange.com/a/2610/223981 that fixes a spacing issue with \left\right
\let\originalleft\left
\let\originalright\right
\renewcommand{\left}{\mathopen{}\mathclose\bgroup\originalleft}
\renewcommand{\right}{\aftergroup\egroup\originalright}

\title{15-Theorem and Its Generalizations: A Journey from Ramanujan to Conway and Bhargava\footnote{Fall 2024 Math 254A Term Paper. I wrote this myself, with help from Google Gemini AI for generating citations.}: Presentation Edition}
\author{Mingyang Cen \\ Student ID: 3036462810 \\ Email address: \texttt{mcen@berkeley.edu}}

\begin{document}

\maketitle

\section{Introduction}
Lagrange's famous four-square theorem states that the quadratic form $x^2+y^2+z^2+w^2$ represents all natural numbers. One may wonder which other quadratic forms have this property of representing all natural numbers. For example, $x^2 + y^2 + z^2 + 2w^2$ works as well. A solution is provided by the 15-Theorem, which I will discuss in this talk.

The 15-Theorem states that a positive-definite integer-matrix quadratic form (which I will just call a quadratic form from now on) represents all positive integers exactly when it represents 1, 2, 3, 5, 6, 7, 10, 14, and 15. These nine numbers are called the critical numbers.

A useful corollary of the 15-Theorem is that there are exactly 204 universal four-dimensional quadratic forms.

\section{Definitions}
Here is some machinery we need for the proof.

There is a well-known natural bijection between lattices and quadratic forms, so any operation on a lattice can be applied to a quadratic form, and vice versa. I am going to use both lattices and quadratic forms in the proof, whichever is most convenient at the time.

A quadratic form $Q$ is defined to be \emph{universal} if it represents all positive integers. If $Q$ is not universal, its \emph{truant} is defined to be the smallest positive integer not represented. An \emph{escalation} of a lattice $L$ is a lattice generated by $L$ and a vector of norm equal to the truant of $L$. An \emph{escalator lattice} is one reached from a series of escalations from the zero lattice. Escalator lattices contain all the necessary information for any quadratic form, so we only have to look at escalators.

\section{Proof of the 15-Theorem}
\subsection{Escalators of dimension 1, 2, and 3}
It is very easy to determine the escalators of dimension up to 4, but when we try to escalate again, we find that some of the 207 four-dimensional escalators are universal.

\subsection{Escalators of dimension 4}
We want to show 201 of the 207 quaternary escalators are universal, and the other 6 have truants 10 or 15.

As an illustrative example for the argument for the 201 universal quaternaries, let $L$ be an escalation of $L' = \begin{bsmallmatrix} 1 & 0 & 0 \\ 0 & 2 & 0 \\ 0 & 0 & 2 \end{bsmallmatrix}$, where the orthogonal complement of $L'$ in $L$ has Gram matrix with one entry $m$. Since $L'$ is unique in its genus, we may apply the Hasse-Minkowski principle to go from the global setting of the integers to the local setting of the $p$-adic numbers. So, $L'$ represents everything but positive integers of the form $4^a (8b + 7)$, where $a$ and $b$ are nonnegative integers.

Suppose for the sake of contradiction that $L$ is not universal, and let its truant be $u$. Through some argument, we find that $u$ is 7 modulo 8.

Using arguments on $m \pmod 8$, we obtain three possibilities:
\begin{itemize}
    \item $u < 4m$
    \item $u - m$ represented by $L'$
    \item $u - 4m$ represented by $L'$
\end{itemize}
Because $L = L' \oplus [m]$, if $u - m$ or $u - 4m$ is represented by $L'$, $u$ is represented by $L$. This is false by assumption. Hence, $u$ must be less than $4m$. By computation, $m \le 28$, so $u < 112 = 4 \cdot 28$. Now that we have bounded $u$, we use computer checks to show that $L$ represents all positive integers up to 112. This shows $u$ did not exist in the first place, so $L$ is universal.

There are a few special cases where a slightly modified argument must be applied, but in the end, 201 quaternary escalators are shown to be universal, and all five-dimensional escalators are universal. By analyzing each of these escalators and their truants, we prove the 15-Theorem.

\section{Related results}
A theorem called the 290-Theorem provides a similar criterion as the 15-Theorem, for more general quadratic forms.

Several generalizations of the 15-Theorem and the 290-Theorem exist to quadratic forms representing odd numbers, prime numbers, numbers coprime to a certain number, and many others.

\end{document}
