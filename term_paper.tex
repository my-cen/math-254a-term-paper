\documentclass[letterpaper, 12pt]{article}
\usepackage{amsmath, amssymb, mathtools, array} % essentials
\usepackage[left=1in,right=1in,top=1in,bottom=1.5in]{geometry}
\renewcommand{\baselinestretch}{1.1}
\usepackage[backend=bibtex]{biblatex}
\addbibresource{term_paper.bib}
%\bibliographystyle{alpha}
\nocite{*}
\usepackage{enumerate}
\usepackage{float}
\usepackage{amsthm}
\newtheorem{thm}{Theorem}
\newtheorem{conj}{Conjecture}
\usepackage{hyperref}
\hypersetup{
    colorlinks=true,
    linkcolor=blue,
    filecolor=blue,
    urlcolor=blue,
    citecolor=blue,
}
%\usepackage{tikz-cd}
%\usepackage{xcolor}
%\usepackage{url}
%\usepackage{graphicx}
%\graphicspath{{./}}
%\usepackage{enumitem}
%\usepackage{comment}

% convenience names
\newcommand{\nequiv}{\not\equiv}
\newcommand{\nsubset}{\not\subset}
\renewcommand{\emptyset}{\varnothing}
\newcommand{\xto}{\xrightarrow}
\newcommand{\injto}{\hookrightarrow}
\newcommand{\surjto}{\twoheadrightarrow}

\newcommand{\diff}{\mathop{}\!d} % as in $\int e^x \diff x$

\renewcommand{\Re}{\operatorname{Re}}
\renewcommand{\Im}{\operatorname{Im}}
\DeclareMathOperator{\Cl}{Cl} % ideal class group
\DeclareMathOperator{\Frac}{Frac}
\DeclareMathOperator{\Tr}{Tr}
\DeclareMathOperator{\Gal}{Gal}
\DeclareMathOperator{\Stab}{Stab}

% code from https://tex.stackexchange.com/a/2610/223981 that fixes a spacing issue with \left\right
\let\originalleft\left
\let\originalright\right
\renewcommand{\left}{\mathopen{}\mathclose\bgroup\originalleft}
\renewcommand{\right}{\aftergroup\egroup\originalright}

\title{15-Theorem and Its Generalizations: A Journey from Ramanujan to Conway and Bhargava\footnote{Fall 2024 Math 254A Term Paper. I wrote this myself, with help from Google Gemini AI for generating citations. \\ Reviewer: Andrew Cheng}}
\author{Mingyang Cen \\ Student ID: 3036462810 \\ Email address: \texttt{mcen@berkeley.edu}}

\begin{document}

\maketitle

\begin{abstract}
    We study the issue of determining whether a quadratic form represents all positive integers, which is solved by Bhargava and Hanke's 15-Theorem and 290-Theorem. We prove the 15-Theorem and a sketch of the 290-Theorem using the pioneering technique in the original paper. The purpose of this paper is to analyze the original papers, add details in some places, and formulate SageMath code for many of the computations within the proof, correcting a possible minor computational error which does not affect the conclusions of the original papers. In addition, we discuss recent similar theorems about quadratic forms representing certain sets of numbers.
\end{abstract}

\tableofcontents

\section{Introduction}
In this paper, a ``quadratic form'' or ``lattice'' refers to one that is positive-definite and integer-matrix, unless otherwise specified. ``Integer-matrix'' means that when the quadratic form is represented as a symmetric matrix, all entries are integers. ``Integer-valued'' is a more lenient condition that allows off-diagonal entries to be half-integers.

A quadratic form is considered to be \emph{universal} if it represents all positive integers.
%
%These theorems have a long history, since the question of which quadratic forms are universal is simple to state but hard to completely solve.
%In the 17th century, Fermat proved the two-square theorem concerning the numbers represented by $x^2 + y^2$.
In 1770, Lagrange \cite{LagrangeFourSquareTheorem} proved his famous four-square theorem, which is equivalent to the statement that $x^2 + y^2 + z^2 + w^2$ is universal. Legendre considered the numbers represented by $x^2 + y^2 + z^2$. During the nineteenth century, Gauss's concept of genus and Hensel's $p$-adic numbers were invented, two crucial tools in the proof. Together, this allows us to show a ternary quadratic form represents a large set of numbers: we only have to look at how it behaves over each ring of $p$-adic integers $\mathbb{Z}_p$.

In 1916, Ramanujan \cite{ramanujan1916diagonal} gave a list of universal diagonal quaternary quadratic forms and claimed that there were no others. This claim was confirmed by Dickson \cite{Dickson1927} in 1927. This motivated research on finding a complete classification of universal quadratic forms.
Willerding \cite{willerding1947} claimed to solve the problem in the case of universal four-dimensional integer-matrix quadratic forms, finding 178 quadratic forms.

In 1993, Conway and Schneeberger announced the first proof of the $15$-Theorem, a theorem giving a full criterion for a quadratic form to be universal. This also provided a shortcut to finding all universal quaternaries, showing Willerding had made a couple of mistakes. There are in fact 204 universal quaternaries.
The proof of the 15-theorem was extremely complex and required separately analyzing every single case, so it is unpublished.
In 2000, Bhargava \cite{fifteen} pioneered a new technique to prove the 15-Theorem, using escalations of lattices, a much simpler, more uniform method. This also showed that Willerding had made a couple of errors. We will present this versatile method in this paper.

In 2011, Bhargava and Hanke \cite{twoninety} used the same technique, combined with modular forms to deal with every case, to successfully prove the 290-Theorem, a similar result on integer-valued quadratic forms.
Using a very similar technique, Rouse proved a ``451-Theorem'' \cite{451} on integer-valued quadratic forms representing all odd positive integers in 2014, assuming certain conjectures. Similar theorems have been proved for different sets of numbers to be represented.

The purpose of this paper is to analyze the original papers and add details for the calculations in some places. In addition, we formulate SageMath code for many of the computations within the proof, correcting a possible minor computational error which does not affect the conclusions of the original papers. We will discuss recent similar theorems about quadratic forms representing certain sets of numbers.

\section{Statements of the Theorems}
In this term paper, we will prove the 15-Theorem and its associated corollaries from \cite{fifteen}, and give a sketch of the 290-Theorem with its corollaries from \cite{twoninety}. Proofs will be given in the next two sections.

\subsection{The 15-Theorem and Related Results}
\begin{thm}[The 15-Theorem]\label{15thm}
    A quadratic form is universal (represents all positive integers) if and only if it represents the nine numbers
    \[1, 2, 3, 5, 6, 7, 10, 14, 15.\]
\end{thm}
Each number here (called a \emph{critical number}) is necessary because of the following result:
\begin{thm}\label{critnumsneeded}
    For each critical number $n$, there exists a quadratic form representing all positive integers except $n$.
\end{thm}
The number 15 is special in this theorem since:
\begin{thm}\label{15thmgeneral}
    A quadratic form representing all critical numbers besides $15$ must represent all positive integers besides $15$.
\end{thm}
By application of the 15-Theorem, we easily obtain an extremely useful classification of universal four-dimensional quadratic forms. These are listed in Appendix A, Table A.6.
\begin{thm}\label{universal 4d}
    There are exactly $204$ non-isomorphic four-dimensional quadratic forms representing all positive integers.
\end{thm}

\subsection{The 290-Theorem and Related Results}\label{section:290thm}
\begin{thm}[The 290-Theorem]\label{290thm}
    A positive-definite \emph{integer-valued} quadratic form represents all positive integers if and only if it represents the twenty-nine numbers
    \[1, 2, 3, 5, 6, 7, 10, 13, 14, 15, 17, 19, 21, 22, 23, 26,\]
    \[29, 30, 31, 34, 35, 37, 42, 58, 93, 110, 145, 203, 290.\]
\end{thm}
The 290-Theorem appears to be similar to the 15-Theorem, and the proof has the same structure. However, the weakened condition of being integer-valued vastly increases the number of cases that must be considered. Because of this, the 290-Theorem requires several analytic results on modular forms for the proof.
Similar generalizations of theorems 2, 3, and 4 are:
\begin{thm}\label{290critnumsneeded}
    For each critical number $n$ in the $290$-theorem, there exists an integer-valued quadratic form representing all positive integers except $n$.
\end{thm}
\begin{thm}\label{290thmgeneral}
    An integer-valued quadratic form representing all critical numbers besides $290$ must represent all positive integers besides $290$.
\end{thm}
\begin{thm}\label{290 universal 4d}
    There are exactly 6436 non-isomorphic quaternary integer-valued quadratic forms representing all positive integers.
\end{thm}

\section{Definitions}
In order to prove the 15-Theorem and 290-Theorem, we will need a few pieces of machinery, which we will introduce in this section.

There is a well-known natural bijection between lattices with integer inner products and positive-definite integer-matrix quadratic forms, so any operation on a lattice can be applied to a quadratic form.

A quadratic form $Q$ or its corresponding lattice is defined to be \emph{universal} if it represents all positive integers. If $Q$ is not universal, its \emph{truant} is defined to be the smallest positive integer not represented by $Q$. An \emph{escalation} of a lattice $L$ is a lattice generated by $L$ and a vector of norm equal to the truant of $L$. An \emph{escalator lattice} is one reached from a series of escalations from the zero lattice.

Here is an example of these concepts. The quadratic form
\[Q(x, y) = x^2 + y^2 = \begin{bmatrix} 1 & 0 \\ 0 & 1 \end{bmatrix}\]
represents $0, 1, 2, 4, 5, 8, 9, 10, \ldots$. Therefore, $Q$ is not universal, and it has truant $3$. One possible escalation would be
\[\begin{bmatrix} 1 & 0 & 0 \\ 0 & 1 & 1 \\ 0 & 1 & 3 \end{bmatrix}.\]
This is equivalently represented as $x^2 + y^2 + 2yz + 3z^2$. This represents
\[0, 1, 2, 3, 4, 5, 6, 7, 8, 9, 10, 12, 13, 16, \ldots\]
A very useful observation is that an escalation $Q'$ of a quadratic form $Q$ must represent the truant of $Q$. Hence, either $Q'$ is universal or the truant of $Q'$ is greater than the truant of $Q$.

\section{Proof of the 15-Theorem}
Within any quadratic form $Q$, we may construct a chain of escalations from the zero lattice, within $Q$. Therefore, we only actually have to prove the 15-Theorem for escalators, which we will do in the following sections.
\subsection{Escalators of Dimension 1, 2, and 3}
The only quadratic form in $0$ variables is $0$, with truant $1$. Its unique escalation is $x^2$ (equivalently $\begin{bmatrix} 1 \end{bmatrix}$ as a matrix) with truant $2$, hence further escalations of dimension $2$ must have Gram matrix of the form
\[\begin{bmatrix} 1 & a \\ a & 2 \end{bmatrix}.\]
By Cauchy-Schwarz, $a$ is $0$ or $\pm 1$, creating two nonisomorphic escalations with Minkowski-reduced Gram matrices
\[\begin{bmatrix} 1 & 0 \\ 0 & 1 \end{bmatrix}, \begin{bmatrix} 1 & 0 \\ 0 & 2 \end{bmatrix}.\]
Their truants are $3$ and $5$, respectively. By a similar argument, we can compute the $9$ three-dimensional escalations: they have Minkowski-reduced Gram matrices
\[\begin{bmatrix} 1 & 0 & 0 \\ 0 & 1 & 0 \\ 0 & 0 & 1 \end{bmatrix}, \begin{bmatrix} 1 & 0 & 0 \\ 0 & 1 & 0 \\ 0 & 0 & 2 \end{bmatrix}, \begin{bmatrix} 1 & 0 & 0 \\ 0 & 1 & 0 \\ 0 & 0 & 3 \end{bmatrix}, \begin{bmatrix} 1 & 0 & 0 \\ 0 & 2 & 0 \\ 0 & 0 & 2 \end{bmatrix}, \begin{bmatrix} 1 & 0 & 0 \\ 0 & 2 & 0 \\ 0 & 0 & 3 \end{bmatrix}, \begin{bmatrix} 1 & 0 & 0 \\ 0 & 2 & 1 \\ 0 & 1 & 4 \end{bmatrix},\]
\[\begin{bmatrix} 1 & 0 & 0 \\ 0 & 2 & 0 \\ 0 & 0 & 4 \end{bmatrix}, \begin{bmatrix} 1 & 0 & 0 \\ 0 & 2 & 1 \\ 0 & 1 & 5 \end{bmatrix}, \begin{bmatrix} 1 & 0 & 0 \\ 0 & 2 & 0 \\ 0 & 0 & 5 \end{bmatrix}.\]
These are also listed in Appendix A, Table A.5.
By computation, they have truants
\[7, 14, 6, 7, 10, 7, 14, 7, 10.\]
Escalating again gives $207$ nonisomorphic four-dimensional escalator lattices. These 207 escalators are listed in Appendix A, Table A.3. An important issue is to determine that the quaternary escalators either have truant in the critical numbers or are universal. We need to determine which of these 207 are universal and which are not, in order to continue.

\subsection{Escalators of Dimension 4}
By the end of this section, we will show that 201 of the 207 quaternary escalators are universal, and the other 6 have truants 10 or 15.

Most of the four-dimensional escalators $L$ can be proven to be universal using the following argument.
We find a $3$-dimensional sublattice $L'$ that is unique in its genus and use local arguments to prove that $L'$ represents all but a certain set of numbers, using the Hasse-Minkowski theorem. The Hasse-Minkowski theorem allows us to transition from local arguments over the $p$-adic numbers, where we can apply Hensel's lemma, to the integers.
%For example, if $L'$ has Gram matrix $\begin{bsmallmatrix} 1 & 0 & 0 \\ 0 & 1 & 0 \\ 0 & 0 & 1 \end{bsmallmatrix}$, then $L'$ represents numbers that are not of the form $4^a b$, where $a \ge 0$ and $b \equiv 7 \pmod 8$.
Let $m$ be the single entry in the Gram matrix of the orthogonal complement of $L'$ in $L$.
Suppose for the sake of contradiction that $L$ is not universal, and let $u$ be its truant.
We can determine $u$ is squarefree: if $u$ is a multiple of $n^2$ for $n > 1$, then $u/n^2$ cannot be represented by $L$, so $u$ is not the truant, a contradiction. By considering $u - xm$ where $x$ depends on $m$ modulo a certain number, we can find an upper bound for $u$.
By manually checking $L$ represents all natural numbers below the upper bound, this shows $u$ cannot exist, so $L$ is universal.

As an example, let $L$ be an escalation of $L' = \begin{bsmallmatrix} 1 & 0 & 0 \\ 0 & 2 & 0 \\ 0 & 0 & 2 \end{bsmallmatrix}$, where the orthogonal complement of $L'$ in $L$ has Gram matrix with one entry $m$. Since $L'$ is unique in its genus, we may calculate the numbers represented by it using calculations over the completions of $\mathbb{Q}$. Over $\mathbb{Z}/8\mathbb{Z}$, $L'$ represents everything that is not $7$ modulo $8$. Over any odd prime $p$, for any $k \in \mathbb{Z}/p\mathbb{Z}$, we can use the following argument to show $L'$ represents $k$ modulo $p$. If $k/2$ denotes $k$ times the multiplicative inverse of $2$ in $\mathbb{Z}/p\mathbb{Z}$, then the sets $\{x^2 : x \in \mathbb{Z}/p\mathbb{Z}\}$ and $\{k/2 - y^2 : y \in \mathbb{Z}/p\mathbb{Z}\}$ each have cardinality $(p + 1)/2$. Hence, they must intersect, so there exist $x, y \in \mathbb{Z}/p\mathbb{Z}$ such that $k/2 \equiv x^2 + y^2 \pmod p$. Hence, $k \equiv 2x^2 + 2y^2 \pmod p$, proving $L'$ represents $k$ modulo $p$. Because of this, $L'$ represents everything but positive integers of the form $4^a (8b + 7)$, where $a$ and $b$ are nonnegative integers.

Suppose for the sake of contradiction that $L$ is not universal, and let its truant be $u$. Since $L'$ cannot represent $u$, we can write $u$ in the form $4^a (8b + 7)$. Since $u$ is minimal, it is squarefree, so $a = 0$ and $u \equiv 7 \pmod 8$. We consider three cases depending on $m$ modulo $8$:
\begin{itemize}
    \item If $m$ is not congruent to $0$, $3$, or $7$ modulo $8$, then $u-m$ cannot be of the form $4^a (8b + 7)$.
    \item If $m \equiv 3, 7 \pmod 8$, then $u-4m$ cannot be of the form $4^a (8b + 7)$ either.
    \item If $m \equiv 0 \pmod 8$, then $L$ is called exceptional, and we have to use a different $L'$. More details are provided later in this paper.
\end{itemize}

Therefore, either $u < 4m$ or $u - m$ or $u - 4m$ (depending on $m \pmod 8$) is represented by $L'$. Because $L = L' \oplus [m]$, if $u - m$ or $u - 4m$ is represented by $L'$, $u$ is represented by $L$. This is false by assumption. Hence, $u$ must be less than $4m$. By computation, $m \le 28$, so $u < 112 = 4 \cdot 28$. Now that we have bounded $u$, we use computer checks to show that $L$ represents all positive integers up to $112$. This shows $u$ did not exist in the first place, so $L$ is universal.

By a similar argument, we find that most of the $207$ quaternary lattices are universal. The numbers are provided in Appendix A, Table A.1.

\subsection{Special Cases for Certain Matrices}
The argument in the previous section works well most of the time. There are only a couple of cases where different arguments have to be used. The three-dimensional escalator lattice
\[L = \begin{bmatrix} 1 & 0 & 0 \\ 0 & 2 & 1 \\ 0 & 1 & 4 \end{bmatrix}\]
is not unique in its genus.
In general, dealing with a quadratic form like this is extremely difficult. For example, the odd numbers not represented by Ramanujan's ternary quadratic form $x^2 + y^2 + 10z^2$ do not have any predictable pattern \cite{OnoSoundararajan1997}.
In our case, however, $L$ happens to be ``nearly'' unique in its genus. Luckily, $L$ contains the lattice
\[L' = \begin{bmatrix} 1 & 0 & 0 \\ 0 & 4 & 2 \\ 0 & 2 & 8 \end{bmatrix},\]
which is unique in its genus, and $L$ contains the two lattices
\[L''_1 = \begin{bmatrix} 2 & -2 & 2 \\ -2 & 5 & 2 \\ 2 & 2 & 8 \end{bmatrix}, L''_2 = \begin{bmatrix} 3 & 0 & 0 \\ 0 & 5 & 4 \\ 0 & 4 & 5 \end{bmatrix}\]
that together form a genus. The quadratic form $L'$ represents all numbers locally represented by $L'$ that are not congruent to $2, 3 \pmod 4$, and $L''_1, L''_2$ is the same except the condition is replaced with being congruent to $1 \pmod 3$. By the Chinese Remainder Theorem, this makes $L$ represent all numbers it locally represents that are not $7$ or $10 \pmod{12}$. This condition is nearly as good as being unique in a genus, so we may apply essentially the same argument as before.

Exceptional four-dimensional lattices $L$ can usually still have the argument in the previous section apply by changing $L'$. That is, $L$ can be an escalator of both $L'$ and $L''$, but $L$ over $L'$ is exceptional, while $L$ over $L''$ is not exceptional. There are only five cases where $L$ is actually exceptional, that is, there are no three-dimensional escalators $L'$ that make $L$ non-exceptional. In these cases, we find a three-dimensional sublattice $L'$ that is unique in its genus, even though $L'$ might not be an escalator. We can apply mostly the same argument as in the previous section. The matrices used are listed in Appendix A, Table A.2.

\subsection{Escalators of Dimension 5}\label{section:5 dim escalators}
The $6$ nonuniversal four-dimensional escalators are
\[\begin{bmatrix} 1 & 0 & 0 & 0 \\ 0 & 2 & 0 & 1 \\ 0 & 0 & 3 & 0 \\ 0 & 1 & 0 & 4 \end{bmatrix}, \begin{bmatrix} 1 & 0 & 0 & 0 \\ 0 & 2 & 1 & 0 \\ 0 & 1 & 4 & 1 \\ 0 & 0 & 1 & 5 \end{bmatrix}, \begin{bmatrix} 1 & 0 & 0 & 0 \\ 0 & 2 & 1 & 0 \\ 0 & 1 & 5 & 1 \\ 0 & 0 & 1 & 5 \end{bmatrix}, \begin{bmatrix} 1 & 0 & 0 & 0 \\ 0 & 2 & 0 & 0 \\ 0 & 0 & 5 & 0 \\ 0 & 0 & 0 & 5 \end{bmatrix}, \begin{bmatrix} 1 & 0 & 0 & 0 \\ 0 & 2 & 0 & 1 \\ 0 & 0 & 5 & 2 \\ 0 & 1 & 2 & 8 \end{bmatrix},\]
and
\[\begin{bmatrix} 1 & 0 & 0 & 0 \\ 0 & 2 & 0 & 1 \\ 0 & 0 & 5 & 1 \\ 0 & 1 & 1 & 9 \end{bmatrix}.\]
The original paper appears to have a small mistake: the third matrix was mistakenly written as
\[\begin{bmatrix} 1&0&0&0 \\ 0&2&\fbox0&0 \\ 0&\fbox0&5&1 \\ 0&0&1&5\end{bmatrix}\]
instead of
\[\begin{bmatrix} 1&0&0&0 \\ 0&2&\fbox1&0 \\ 0&\fbox1&5&1 \\ 0&0&1&5\end{bmatrix}.\]
These six $4$-dimensional escalators are not universal, with truants $10,10,15,15,15,15$. These six matrices are also listed in Appendix A, Table A.4. It is impractical to consider every single one of the 1630 five-dimensional escalators, but we have a trick. We may apply the same argument as above, but the final check that $L$ represents all positive integers up to some number fails with just one number. This shows that these $6$ lattices each miss that number and no others. That number is the lattice's truant, so the $5$-dimensional escalators of the lattice must be universal. This finishes our analysis of escalators.

\subsection{Final Proof of the 15-Theorem and Related Results}
This shows all escalators of the zero lattice are at most $5$-dimensional, and that there are exactly $1850$ nonisomorphic escalator lattices of the zero lattice: $1$ of dimension $0$, $1$ of dimension $1$, $2$ of dimension $2$, $9$ of dimension $3$, $207$ of dimension $4$, and $1630$ of dimension $5$. This also proves that a universal lattice $L$ must contain a universal sublattice of dimension $5$ or less, because $L$ must contain a chain of escalations of the zero lattice.

\begin{proof}[Proof of the 15-Theorem (Theorem \ref{15thm})]
    Let $Q$ be a quadratic form. Suppose that $Q$ represents all critical numbers. Then, we may construct an escalator lattice $Q'$ within $Q$, such that $Q'$ represents all critical numbers. This is where our analysis of escalators come in handy.
    Each escalator lattice we found is either universal or has truant in the critical numbers. Due to $Q'$ representing all critical numbers, $Q'$ must in fact be universal, so $Q$ is universal. This proves the $15$-Theorem.
\end{proof}
\begin{proof}[Proof of Theorem \ref{critnumsneeded}]
    Let $n$ be a critical number. Let $L$ be a lattice with truant $n$. Then, by the four-square theorem,
    \[L \oplus [n + 1] \oplus [n + 1] \oplus [n + 1] \oplus [n + 1] \oplus [2n + 1]\]
    represents all positive integers except $n$.
\end{proof}
\begin{proof}[Proof of Theorem \ref{15thmgeneral}]
    Since each of the four escalators of the zero lattice that has truant $15$ is four-dimensional and hence must represent all other positive integers, the strengthening of the $15$-Theorem is proved.
\end{proof}

\begin{proof}[Proof of Theorem \ref{universal 4d}]
    The 15-Theorem provides a shortcut to classifying all universal $4$-variable positive-definite integer-matrix quadratic forms. We find a bound on the discriminant of such a quadratic form and systematically apply the $15$-Theorem in each case to find there are exactly $204$ of them. This was used by Bhargava to determine that Willerding's classification of universal quaternary forms had errors.
\end{proof}

\section{A Sketch of the Proof of the 290-Theorem}
\subsection{Introduction and 3-Dimensional Escalators}
The $290$-Theorem \cite{twoninety} states that an integer-\emph{valued} positive-definite quadratic form is universal if and only if it represents the $29$ critical numbers listed in section \ref{section:290thm},
and it has a similar strengthening as the $15$-Theorem.
The structure of the proof is mostly the same.
In this section, a lattice will refer to an integer-\emph{valued} positive-definite lattice.

Through calculation, the $34$ three-dimensional escalators of the zero lattice are determined and calculated to have critical truants.
Their escalations are called \emph{basic} $4$-dimensional escalators. Although many of the basic $4$-dimensional escalators can be escalated to obtain more $4$-dimensional lattices, it suffices to look at only the basic ones.

\subsection{Techniques for Dealing with All Basic 4-Dimensional Escalators}
Many basic four-dimensional escalator lattices $L_4$ can be proven to be universal by the same algebraic argument as in the $15$-Theorem:
find a three-dimensional sublattice $L_3$ unique in its genus that represents a large set of integers, ensure $L_4$ is not exceptional over $L_3$, and use this to show $L_4$ is universal.

However, more than $2300$ of the $6560$ basic four-dimensional escalator lattices cannot be proven to be universal using this method, nor any of the clever tricks for the 15-Theorem. We need a systematic method of treating all these quadratic forms, that does not require them to be special.
Instead, we use modular forms to efficiently treat every one of these quadratic forms. For a quadratic form $Q$ in $n$ variables where $r_Q(m)$ denotes the number of ways $m$ can be represented by $Q$, we may create its Fourier series generating function
\[\Theta_Q(z) = \sum_{m = 0}^{\infty} r_Q(m) e^{2\pi i m z},\]
which turns out to be a weight-$n/2$ modular form. Through a detailed study of the properties of $\Theta_Q$ and several results on modular forms, we find that a quadratic form represents all numbers it locally represents with exceptions only up to a certain small number $N$, which can be computed sufficiently fast. This is nearly the same as the previous condition of being unique in a genus.

In the end, this separates all basic four-dimensional escalator lattices into three categories:
\begin{itemize}
    \item Type I: universal
    \item Type II: not universal but misses at most three values, that are each critical numbers
    \item Type III: represents all positive integers not of the form $4^a(16k+14)$
\end{itemize}
We would like to consider the further escalations of these basic quaternaries and their truants.
For types I and II, this is easy.
Namely, Type I lattices cannot be escalated further at all, and Type II lattices can be escalated at most three times, with each further escalator having truant being one of the critical numbers.
The main problem is considering the escalations of Type III lattices, since they could miss infinitely many numbers. Without further argument, they could conceivably be escalated infinitely many times, yielding quadratic forms of arbitrarily large truant.

The crucial technique used to solve this issue is the ``10-14 switch''.
It is impractical to consider every four- or five-dimensional escalation of a basic Type III escalator lattices individually.
The 10-14 switch provides a uniform way to treat these.
Each of the five Type III basic quaternaries, namely,
\[\begin{bmatrix} 1 & 0 & -1/2 & -3 \\ 0 & 2 & 1 & 0 \\ -1/2 & 1 & 5 & 1 \\ -3 & 0 & 1 & 10 \end{bmatrix}, \begin{bmatrix} 1 & 0 & -1/2 & -2 \\ 0 & 2 & 1 & -2 \\ -1/2 & 1 & 5 & 3 \\ -2 & -2 & 3 & 10 \end{bmatrix},\]
\[\begin{bmatrix} 1 & 0 & -1/2 & -2 \\ 0 & 2 & 1 & -2 \\ -1/2 & 1 & 5 & 1 \\ -2 & -2 & 1 & 10 \end{bmatrix}, \begin{bmatrix} 1 & 0 & -1/2 & -1 \\ 0 & 2 & 1 & 0 \\ -1/2 & 1 & 5 & 3 \\ -1 & 0 & 3 & 10 \end{bmatrix}, \begin{bmatrix} 1 & 0 & -1/2 & -1 \\ 0 & 2 & 1 & 0 \\ -1/2 & 1 & 5 & 2 \\ -1 & 0 & 2 & 10 \end{bmatrix},\]
are of truant $14$, and they are each an escalation of the $3$-dimensional lattice
\[L = \begin{bmatrix} 1 & 0 & 1/2 \\ 0 & 2 & 1 \\ 1/2 & 1 & 5 \end{bmatrix}\]
of truant $10$.
That is, if $L'$ denotes one of the five Type III escalator lattices and $M$ is an escalation of $L'$, then $L'$ is obtained by joining a vector of norm $10$ to $L$, and $L''$ is obtained by joining a vector of norm $14$ of $L'$.

However, it is possible to switch the order of these two operations.
Let an \emph{auxiliary quaternary} be a lattice generated by $L$ and a vector of norm $14$.
Each auxiliary quaternary has truant $10$, so $M$ can be obtained as an escalation of an auxiliary quaternary.
Hence, it suffices to scrutinize the escalations of auxiliary quaternaries, in place of the difficult Type III basic quaterneries.

Impressively, $226$ of the $330$ auxiliary quaternaries are isometric to Type I or Type II basic quaternaries, so they can be escalated at most three times, and their escalators' truants are automatically critical numbers. The other $104$ are either type I, type II, or a new type called Type IV:
\begin{itemize}
    \item Type IV: represents all integers that are not of the form $10n^2$ or $13n^2$.
\end{itemize}
Type I and Type II auxiliary quaternaries already automatically satisfy the desired properties, and Type IV auxiliary quaternaries $Q$ satisfy them because $Q \oplus [10] \oplus [13]$ must be universal.

\subsection{Final Proof of the 290-Theorem and Related Results}
This proves similar results as before. Through inspection of each lattice, the zero lattice cannot be escalated more than $7$ times.

\begin{proof}[Proof of the 290-Theorem]
    By considering chains of escalations within a lattice, all information about lattice universality is contained within the study of escalator lattices. Since all escalator lattices have truants that are critical numbers, the $290$-Theorem is proved.
\end{proof}

Theorem \ref{290critnumsneeded} has mostly the exact same proof as theorem \ref{critnumsneeded}.

\begin{proof}[Proof of theorem \ref{290thmgeneral}]
    We notice that all escalators of truant $290$ must be a sequence of escalations of
    \[\begin{bmatrix}
    1 & 0 & -1/2 & 0 \\
    0 & 2 & -1/2 & 0 \\
    -1/2 & -1/2 & 4 & 0 \\
    0 & 0 & 0 & 29
    \end{bmatrix},\]
    which fails to represent precisely $145, 203, 290$. Since each of these numbers is less than $290$, this proves the generalization.
\end{proof}

\begin{proof}[Proof of theorem \ref{290 universal 4d}]
    We use the same argument as for the case of integer-matrix quadratic forms: finding a bound on the discriminant of such a quadratic form, then systematically applying the 15-Theorem through each case.
\end{proof}

\section{Related Results}

\begin{table}[H]
    \begin{tabular}{|c|c|c|}
        \hline
        & Integer-Matrix & Integer-Valued \\ \hline
        Universal & 15-Theorem & 290-Theorem \\
        (all positive integers) && \\ \hline
        Odd universal & 33-Theorem & 451-Theorem \\
        (all odd positive integers) && \\ \hline
        Coprime universal & Lists in table \ref{coprimeunivtable} & \\
        (all positive integers && \\
        coprime to a certain prime) && \\ \hline
    \end{tabular}
    \caption{Theorems related to the 15-Theorem}
\end{table}
\begin{conj}\label{451thmconj}
    The three ternary quadratic forms
    \begin{align*}
        & x^2 + 2y^2 + 5z^2 + xz \\
        & x^2 + 3y^2 + 6z^2 + xy + 2yz \\
        & x^2 + 3y^2 + 7z^2 + xy + xz
    \end{align*}
    are odd universal (represent all positive odd integers).
\end{conj}
\begin{thm}[The 451-Theorem]
    Assuming Conjecture \ref{451thmconj}, a positive-definite integer-valued quadratic form represents all positive \emph{odd} integers if and only if it represents the $46$ numbers
    \[1, 3, 5, 7, 11, 13, 15, 17, 19, 21, 23, 29, 31, 33, 35, 37, 39, 41, 47, 51, 53, 57, 59, 77,\]
    \[83, 85, 87, 89, 91, 93, 105, 119, 123, 133, 137, 143, 145,\]
    \[187, 195, 203, 205, 209, 231, 319, 385, 451.\]
    For each of these numbers $n$, there exists a positive-definite integer-valued quadratic form representing all positive odd integers besides $n$. Also, a positive-definite integer-valued quadratic form representing each of the $46$ critical numbers besides $451$ must represent all positive odd integers besides $451$.
\end{thm}
\begin{thm}[The 33-Theorem]
    A positive-definite integer-valued quadratic form represents all positive odd integers if and only if it represents
    \[1, 3, 5, 7, 11, 15, 33.\]
\end{thm}

The $451$-Theorem uses the same proof technique as the $290$-Theorem and can be generalized in the same way. It has a similar special case for integer-matrix quadratic forms, namely, the $33$-Theorem.

However, there exist odd universal integer-valued positive-definite quadratic forms that are only three-dimensional. Kaplansky \cite{IrvingKaplansky1995} proved there are at most 23 such quadratic forms, with 19 confirmed. Jagy \cite{Jagy1996FiveRO} proved another one is odd universal, but there are still the three quadratic forms of Conjecture \ref{451thmconj}. The $451$-Theorem relies on these three being odd universal, which has not been proved yet, although it is true under the Generalized Riemann Hypothesis. There have been various generalizations to quadratic forms representing prime numbers \cite{primeuniversaloeis}, numbers coprime to a certain prime \cite{coprimeuniv}, and more. In fact, Bhargava has determined that for every set of positive integers $S$, there is a finite subset $T$ of critical numbers such that $S$ and $T$ obey a sort of 290- or 15-Theorem.

Recently, Matteo Bordignon and Giacomo Cherubini have proved a generalized version of these theorems, for quadratic forms representing all numbers coprime to a certain prime ($p$-coprime universal). Assuming the Generalized Riemann Hypothesis and if $S$ and $S_2$ denote the sets of critical numbers for the $290$-Theorem and $451$-Theorem, the set of critical numbers for $p$-coprime universality is:
\begin{table}[H]
    \begin{tabular}{|c|l|l|} \hline
        $p$ & Integer-matrix critical numbers & Integer-valued \\ \hline
        $1$ & $S$ & $1, 2, 3, 5, 6, 7, 10, 14, 15$ \\ \hline
        $2$ & $S_2$ & 1, 3, 5, 7, 11, 15, 33 \\ \hline
        $3$ & $S \cup \{11, 38, 46, 47,$ &  1, 2, 5, 7, 10, 11, \\
        & $55, 62, 70, 94, 119\} - 3\mathbb{Z}$ & 14, 19, 22, 31, 35 \\ \hline
        $5$ & $S \cup \{38, 39, 46, 47,$ & 1, 2, 3, 6, 7, 13, \\
        & $53, 61, 62, 74, 78\} - 5\mathbb{Z}$ & 14, 21, 26, 29, 58 \\ \hline
        $7$ & $S \cup \{39, 46, 47, 55, 62, 78, 142\} - 7\mathbb{Z}$ & 1, 2, 3, 5, 6, 10, 15, 23, \\
        && 30, 31, 39, 55, 78 \\ \hline
        $\ge 11$ & $S - p\mathbb{Z}$ & Same as 15-Theorem \\ \hline
    \end{tabular}
    \caption{Critical numbers for $p$-coprime universal quadratic forms}
    \label{coprimeunivtable}
\end{table}

\section{Methodology}
The source code for this term paper is at:\\\url{https://github.com/my-cen/math-254a-term-paper/blob/main/term_paper_sagemath.ipynb}.
I have calculated the nonuniversal quaternary and ternary escalators, together with their truants, and found a possible minor error in the original paper for the list of $6$ nonuniversal quaternary escalators, mentioned above in section \ref{section:5 dim escalators}.

\section{Acknowledgements}
I thank Andrew Cheng for many comments and criticisms to this term paper.
\fontsize{10}{12}\selectfont

\section{Bibliography}
\printbibliography[heading=none]
\section{Appendix A: Tables of Quadratic Forms}

{
% TeX magic to turn tables into A.1,A.2,A.3 format
\setcounter{table}{0}
\renewcommand{\thetable}{A.\arabic{table}}

\begin{table}[ht]
    \centering
    \begin{tabular}{|c|c|c|c|c|c|c|}
    \hline
     & 3D Escalator & Truant & Nos. not repr\footnotemark & If $m$ & Subtract & Check up to \\ \hline
    (1) & $\begin{bmatrix} 1 & 0 & 0 \\ 0 & 1 & 0 \\ 0 & 0 & 1 \end{bmatrix}$ & 7 & $2^e u_7$ & $\not\equiv 0 \pmod{8}$ & $m$ or $4m$ & 112 \\
     & & & & $\equiv 0 \pmod{8}$ & does not arise & - \\ \hline
    (2) & $\begin{bmatrix} 1 & 0 & 0 \\ 0 & 1 & 0 \\ 0 & 0 & 2 \end{bmatrix}$ & 14 & $2^d u_7$ & $\not\equiv 0 \pmod{16}$ & $m$ or $4m$ & 224 \\
     & & & & $\equiv 0 \pmod{16}$ & does not arise & - \\ \hline
    (3) & $\begin{bmatrix} 1 & 0 & 0 \\ 0 & 1 & 0 \\ 0 & 0 & 3 \end{bmatrix}$ & 6 & $3^d u_-$ & $\not\equiv 0 \pmod{9}$ & $m, 4m,$ or $16m$ & 864 \\
     & & & & $\equiv 0 \pmod{9}$ & does not arise & - \\ \hline
    (4) & $\begin{bmatrix} 1 & 0 & 0 \\ 0 & 2 & 0 \\ 0 & 0 & 2 \end{bmatrix}$ & 7 & $2^e u_7$ & $\not\equiv 0 \pmod{8}$ & $m$ or $4m$ & 112 \\
     & & & & $\equiv 0 \pmod{8}$ & [See Table 2] & - \\ \hline
    (5) & $\begin{bmatrix} 1 & 0 & 0 \\ 0 & 2 & 0 \\ 0 & 0 & 3 \end{bmatrix}$ & 10 & $2^d u_5$ & $\not\equiv 0 \pmod{16}$ & $m$ or $4m$ & 1440 \\
     & & & & $\equiv 0 \pmod{16}$ & does not arise & - \\ \hline
    (6) & $\begin{bmatrix} 1 & 0 & 0 \\ 0 & 2 & 1 \\ 0 & 1 & 4 \end{bmatrix}$ & 7 & $7^d u_-$ & $\not\equiv 0, 3, 9 \pmod{12}$ & $m, 4m,$ or $9m$ & 3087 \\
    & & & & $\& \not\equiv 0 \pmod{49}$ & & \\
     & & & or $7, 10 \pmod{12}$ & $\equiv 0 \pmod{49}$ & does not arise & - \\
     & & & & $\equiv 0, 3, 9 \pmod{12}$ & [See Table 2] & - \\ \hline
    (7) & $\begin{bmatrix} 1 & 0 & 0 \\ 0 & 2 & 0 \\ 0 & 0 & 4 \end{bmatrix}$ & 14 & $2^d u_7$ & $\not\equiv 0 \pmod{16}$ & $m$ or $4m$ & 224 \\
     & & & & $\equiv 0 \pmod{8}$ & [See Table 2] & - \\ \hline
    (8) & $\begin{bmatrix} 1 & 0 & 0 \\ 0 & 2 & 1 \\ 0 & 1 & 5 \end{bmatrix}$ & 7 & $2^e u_7$ & $\not\equiv 0 \pmod{8}$ & $m$ or $4m$ & 252 \\
     & & & & $\equiv 0 \pmod{8}$ & does not arise & - \\ \hline
    (9) & $\begin{bmatrix} 1 & 0 & 0 \\ 0 & 2 & 0 \\ 0 & 0 & 5 \end{bmatrix}$ & 10 & $5^d u_-$ & $\not\equiv 0 \pmod{25}$ & $m$ or $4m$ & 4000 \\
     & & & & $\equiv 0 \pmod{25}$ & does not arise & - \\ \hline
    \end{tabular}
    \caption{Proving universality of 4D escalators in nonexceptional cases \cite{fifteen}}
    \label{tab:my_table}
\end{table}
\footnotetext{$p^d$ denotes an odd power of $p$, $p^e$ denotes an even power of $p$, $u_1,u_3,u_5,u_7$ are $1,3,5,7$ modulo $8$, $u_+,u_-$ are quadratic residues/non-residues modulo $p$}

\begin{table}[ht]
    \centering
    \begin{tabular}{|c|c|p{1in}|c|c|c|}
    \hline
    Exceptional Lattice & New unique-in-genus lattice & Nos. not repr. & $m$ & Subtract & Check up to \\
    \hline
    $\begin{bmatrix} 1 & 0 & 0 & 2 \\ 0 & 2 & 0 & 1 \\ 0 & 0 & 2 & 1 \\ 2 & 1 & 1 & 7 \end{bmatrix}$ & $\begin{bmatrix} 1 & 0 & 0 \\ 0 & 2 & 1 \\ 0 & 1 & 3 \end{bmatrix}$ & $5^d u_+$ & 40 & $m$ or $4m$ & 160 \\
    \hline
    $\begin{bmatrix} 1 & 0 & 0 & 0 \\ 0 & 2 & 0 & 1 \\ 0 & 0 & 2 & 1 \\ 0 & 1 & 1 & 7 \end{bmatrix}$ & $\begin{bmatrix} 2 & 0 & 1 \\ 0 & 2 & 1 \\ 1 & 1 & 7 \end{bmatrix}$ & $2^e u_1, 2^e u_5,$ \newline $2^d u_3, 2^d u_7, 3^d u_+$ & 1 & $m$ & 14 \\ \hline
    $\begin{bmatrix} 1 & 0 & 0 & 1 \\ 0 & 2 & 1 & 0 \\ 0 & 1 & 4 & 3 \\ 1 & 0 & 3 & 7 \end{bmatrix}$ & $\begin{bmatrix} 2 & 1 & 1 \\ 1 & 4 & 0 \\ 1 & 0 & 4 \end{bmatrix}$ & $2^d u_7$ & 1 & $m, 4m, \text{or } 9m$ & 9 \\
    \hline
    $\begin{bmatrix} 1 & 0 & 0 & 0 \\ 0 & 2 & 1 & 1 \\ 0 & 1 & 4 & 0 \\ 0 & 1 & 0 & 7 \end{bmatrix}$ \footnotemark & $\begin{bmatrix} 2 & 0 & 0 \\ 0 & 4 & 2 \\ 0 & 2 & 10 \end{bmatrix}$ & $2^d u_7$ & 90 & $m \text{ or } 4m$ & 504 \\
    \hline
    $\begin{bmatrix} 1 & 0 & 0 & 1 \\ 0 & 2 & 0 & 0 \\ 0 & 0 & 4 & 2 \\ 1 & 0 & 2 & 14 \end{bmatrix}$ & $\begin{bmatrix} 1 & 0 & 0 \\ 0 & 4 & 2 \\ 0 & 2 & 13 \end{bmatrix}$ & $2^d u_5, 2^e u_3$ & 2 & $m \text{ or } 4m$ & 8 \\
    \hline
    \end{tabular}
    \caption{Proving universality of 4D escalators in exceptional cases \cite{fifteen}}
\end{table}
\footnotetext{This only shows even numbers are represented, but the original argument in Table 1 shows odd numbers are always represented.}
\clearpage

\begin{verbatim}
Table A.3: The 207 four-dimensional escalators
where D:,a,b,c,d,e,f denotes the quadratic form
[ 1  0   0   0  ]
[ 0  a  f/2 e/2 ]
[ 0 f/2  b  d/2 ]
[ 0 e/2 d/2  c  ]
where
[  a  f/2 e/2 ]
[ f/2  b  d/2 ]
[ e/2 d/2  c  ]
has determinant D: [1]
----------------------
1:,1,1,1,0,0,0
2:,1,1,2,0,0,0
3:,1,1,3,0,0,0
3:,1,2,2,2,0,0
4:,1,1,4,0,0,0
4:,1,2,2,0,0,0
4:,2,2,2,2,2,0
5:,1,1,5,0,0,0
5:,1,2,3,2,0,0
6:,1,1,6,0,0,0
6:,1,2,3,0,0,0
6:,2,2,2,2,0,0
7:,1,1,7,0,0,0
7:,1,2,4,2,0,0
7:,2,2,3,2,0,2
8:,1,2,4,0,0,0
8:,1,3,3,2,0,0
8:,2,2,2,0,0,0
8:,2,2,3,2,2,0
9:,1,2,5,2,0,0
9:,1,3,3,0,0,0
9:,2,2,3,0,0,2
10:,1,2,5,0,0,0
10:,2,2,3,2,0,0
10:,2,2,4,2,0,2
11:,1,2,6,2,0,0
11:,1,3,4,2,0,0
12:,1,2,6,0,0,0
12:,1,3,4,0,0,0
12:,2,2,3,0,0,0
12:,2,2,4,0,0,2
13:,2,2,5,2,0,2
13:,2,3,3,2,2,0
14:,1,2,7,0,0,0
14:,1,3,5,2,0,0
14:,2,2,4,2,0,0
15:,1,2,8,2,0,0
15:,1,3,5,0,0,0
15:,2,2,5,0,0,2
15:,2,3,3,0,2,0
16:,1,2,8,0,0,0
16:,2,2,4,0,0,0
16:,2,3,3,2,0,0
17:,1,2,9,2,0,0
17:,1,3,6,2,0,0
17:,2,3,4,0,2,2
18:,1,2,9,0,0,0
18:,1,3,6,0,0,0
18:,2,2,5,2,0,0
18:,2,3,3,0,0,0
18:,2,3,4,2,0,2
19:,1,2,10,2,0,0
19:,2,3,4,2,2,0
20:,1,2,10,0,0,0
20:,2,2,5,0,0,0
20:,2,2,6,2,2,0
20:,2,4,4,4,2,0
21:,2,3,4,0,2,0
22:,1,2,11,0,0,0
22:,2,2,6,2,0,0
22:,2,3,4,2,0,0
22:,2,3,5,0,2,2
23:,1,2,12,2,0,0
23:,2,3,5,2,0,2
24:,1,2,12,0,0,0
24:,2,2,6,0,0,0
24:,2,2,7,2,2,0
24:,2,3,4,0,0,0
24:,2,4,4,0,2,2
24:,2,4,4,4,0,0
25:,1,2,13,2,0,0
25:,2,3,5,2,2,0
26:,1,2,13,0,0,0
26:,2,2,7,2,0,0
26:,2,4,4,2,2,0
27:,1,2,14,2,0,0
27:,2,3,5,0,2,0
27:,2,4,5,4,0,2
28:,1,2,14,0,0,0
28:,2,2,7,0,0,0
28:,2,3,5,2,0,0
28:,2,4,4,0,2,0
28:,2,4,5,4,2,0
30:,2,3,5,0,0,0
30:,2,4,4,2,0,0
31:,2,3,6,2,2,0
31:,2,4,5,0,2,2
32:,2,4,4,0,0,0
32:,2,4,5,4,0,0
33:,2,3,6,0,2,0
33:,2,4,5,2,0,2
34:,2,3,6,2,0,0
34:,2,4,5,2,2,0
34:,2,4,6,4,0,2
35:,2,4,5,0,0,2
36:,2,3,6,0,0,0
36:,2,4,5,0,2,0
36:,2,4,6,4,2,0
36:,2,5,5,4,2,2
37:,2,5,5,4,2,0
38:,2,4,5,2,0,0
38:,2,4,6,0,2,2
39:,2,3,7,0,2,0
40:,2,3,7,2,0,0
40:,2,4,5,0,0,0
40:,2,4,6,2,0,2
40:,2,4,6,4,0,0
41:,2,4,7,4,0,2
42:,2,3,7,0,0,0
42:,2,4,6,0,0,2
42:,2,4,6,2,2,0
42:,2,5,5,4,0,0
43:,2,3,8,2,2,0
43:,2,5,5,2,0,2
44:,2,4,6,0,2,0
45:,2,4,7,0,2,2
45:,2,5,5,0,2,0
45:,2,5,6,4,2,2
46:,2,3,8,2,0,0
46:,2,4,6,2,0,0
46:,2,5,6,4,0,2
47:,2,4,7,2,0,2
47:,2,5,6,4,2,0
48:,2,3,8,0,0,0
48:,2,4,6,0,0,0
48:,2,5,5,2,0,0
49:,2,3,9,2,2,0
49:,2,4,7,0,0,2
49:,2,5,6,0,2,2
50:,2,4,7,2,2,0
50:,2,5,5,0,0,0
51:,2,3,9,0,2,0
52:,2,3,9,2,0,0
52:,2,5,6,2,0,2
52:,2,5,6,4,0,0
53:,2,5,6,2,2,0
54:,2,3,9,0,0,0
54:,2,4,7,2,0,0
54:,2,5,6,0,0,2
54:,2,5,7,4,2,2
55:,2,3,10,2,2,0
55:,2,5,6,0,2,0
55:,2,5,7,4,0,2
56:,2,4,7,0,0,0
56:,2,4,8,4,0,0
57:,2,3,10,0,2,0
58:,2,3,10,2,0,0
58:,2,4,8,2,2,0
58:,2,5,6,2,0,0
58:,2,5,7,0,2,2
60:,2,3,10,0,0,0
60:,2,4,9,4,2,0
60:,2,5,6,0,0,0
61:,2,5,7,2,0,2
62:,2,4,8,2,0,0
62:,2,5,7,4,0,0
63:,2,5,7,0,0,2
63:,2,5,7,2,2,0
64:,2,4,8,0,0,0
66:,2,4,9,2,2,0
67:,2,5,8,4,2,0
68:,2,4,9,0,2,0
68:,2,4,10,4,2,0
68:,2,5,7,2,0,0
70:,2,4,9,2,0,0
70:,2,5,7,0,0,0
72:,2,4,9,0,0,0
72:,2,4,10,4,0,0
72:,2,5,8,4,0,0
74:,2,4,10,2,2,0
76:,2,4,10,0,2,0
77:,2,5,9,4,2,0
78:,2,4,10,2,0,0
78:,2,5,8,2,0,0
80:,2,4,10,0,0,0
80:,2,4,11,4,0,0
80:,2,5,8,0,0,0
82:,2,4,11,2,2,0
82:,2,5,9,4,0,0
83:,2,5,9,2,2,0
85:,2,5,9,0,2,0
86:,2,4,11,2,0,0
87:,2,5,10,4,2,0
88:,2,4,11,0,0,0
88:,2,4,12,4,0,0
88:,2,5,9,2,0,0
90:,2,4,12,2,2,0
90:,2,5,9,0,0,0
92:,2,4,13,4,2,0
92:,2,5,10,4,0,0
93:,2,5,10,2,2,0
94:,2,4,12,2,0,0
95:,2,5,10,0,2,0
96:,2,4,12,0,0,0
96:,2,4,13,4,0,0
98:,2,4,13,2,2,0
98:,2,5,10,2,0,0
100:,2,4,13,0,2,0
100:,2,4,14,4,2,0
100:,2,5,10,0,0,0
102:,2,4,13,2,0,0
104:,2,4,13,0,0,0
104:,2,4,14,4,0,0
106:,2,4,14,2,2,0
108:,2,4,14,0,2,0
110:,2,4,14,2,0,0
112:,2,4,14,0,0,0
\end{verbatim}
\clearpage
\begin{verbatim}
Table A.4: Non-universal quaternary integer-matrix escalators [1] (verified with my code)
-----------------------------------------------------------------------------------------
[ 1 0 0 0 ]
[ 0 2 0 1 ]
[ 0 0 3 0 ]
[ 0 1 0 4 ]
Truant: 10

[ 1 0 0 0 ]
[ 0 2 1 0 ]
[ 0 1 4 1 ]
[ 0 0 1 5 ]
Truant: 10

[ 1 0 0 0 ]
[ 0 2 1 0 ]
[ 0 1 5 1 ]
[ 0 0 1 5 ]
Truant: 15

[ 1 0 0 0 ]
[ 0 2 0 0 ]
[ 0 0 5 0 ]
[ 0 0 0 5 ]
Truant: 15

[ 1 0 0 0 ]
[ 0 2 0 1 ]
[ 0 0 5 2 ]
[ 0 1 2 8 ]
Truant: 15

[ 1 0 0 0 ]
[ 0 2 0 1 ]
[ 0 0 5 1 ]
[ 0 1 1 9 ]
Truant: 15
----------
The number of nonuniversal quaternary escalators is 6.
\end{verbatim}
\clearpage
\begin{verbatim}
Table A.5: Truants of three-dimensional integer-matrix escalators. [1] (verified with my code)
----------------------------------------------------------------------------------------------
[ 1 0 0 ]
[ 0 1 0 ]
[ 0 0 1 ]
Truant: 7

[ 1 0 0 ]
[ 0 1 0 ]
[ 0 0 2 ]
Truant: 14

[ 1 0 0 ]
[ 0 1 0 ]
[ 0 0 3 ]
Truant: 6

[ 1 0 0 ]
[ 0 2 0 ]
[ 0 0 2 ]
Truant: 7

[ 1 0 0 ]
[ 0 2 0 ]
[ 0 0 3 ]
Truant: 10

[ 1 0 0 ]
[ 0 2 1 ]
[ 0 1 4 ]
Truant: 7

[ 1 0 0 ]
[ 0 2 0 ]
[ 0 0 4 ]
Truant: 14

[ 1 0 0 ]
[ 0 2 1 ]
[ 0 1 5 ]
Truant: 7

[ 1 0 0 ]
[ 0 2 0 ]
[ 0 0 5 ]
Truant: 10
\end{verbatim}
\clearpage
\begin{verbatim}
Table A.6: The 204 universal quaternaries [1]
---------------------------------------------
1:,1,1,1,0,0,0
2:,1,1,2,0,0,0
3:,1,1,3,0,0,0
3:,1,2,2,2,0,0
4:,1,1,4,0,0,0
4:,1,2,2,0,0,0
4:,2,2,2,2,2,0
5:,1,1,5,0,0,0
5:,1,2,3,2,0,0
6:,1,1,6,0,0,0
6:,1,2,3,0,0,0
6:,2,2,2,2,0,0
7:,1,1,7,0,0,0
7:,1,2,4,2,0,0
7:,2,2,3,2,0,2
8:,1,2,4,0,0,0
8:,1,3,3,2,0,0
8:,2,2,2,0,0,0
8:,2,2,3,2,2,0
9:,1,2,5,2,0,0
9:,1,3,3,0,0,0
9:,2,2,3,0,0,2
10:,1,2,5,0,0,0
10:,2,2,3,2,0,0
10:,2,2,4,2,0,2
11:,1,2,6,2,0,0
11:,1,3,4,2,0,0
12:,1,2,6,0,0,0
12:,1,3,4,0,0,0
12:,2,2,3,0,0,0
12:,2,2,4,0,0,2
12:,2,3,3,0,2,2
13:,2,2,5,2,0,2
13:,2,3,3,2,2,0
14:,1,2,7,0,0,0
14:,1,3,5,2,0,0
14:,2,2,4,2,0,0
15:,1,2,8,2,0,0
15:,1,3,5,0,0,0
15:,2,2,5,0,0,2
15:,2,3,3,0,2,0
16:,1,2,8,0,0,0
16:,2,2,4,0,0,0
16:,2,3,3,2,0,0
17:,1,2,9,2,0,0
17:,1,3,6,2,0,0
17:,2,3,4,0,2,2
18:,1,2,9,0,0,0
18:,1,3,6,0,0,0
18:,2,2,5,2,0,0
18:,2,3,3,0,0,0
18:,2,3,4,2,0,2
19:,1,2,10,2,0,0
19:,2,3,4,2,2,0
20:,1,2,10,0,0,0
20:,2,2,5,0,0,0
20:,2,2,6,2,2,0
20:,2,3,4,0,0,2
20:,2,4,4,4,2,0
22:,1,2,11,0,0,0
22:,2,2,6,2,0,0
22:,2,3,4,2,0,0
22:,2,3,5,0,2,2
23:,1,2,12,2,0,0
23:,2,3,5,2,0,2
24:,1,2,12,0,0,0
24:,2,2,6,0,0,0
24:,2,2,7,2,2,0
24:,2,3,4,0,0,0
24:,2,4,4,0,2,2
24:,2,4,4,4,0,0
25:,1,2,13,2,0,0
25:,2,3,5,0,0,2
25:,2,3,5,2,2,0
26:,1,2,13,0,0,0
26:,2,2,7,2,0,0
26:,2,4,4,2,2,0
27:,1,2,14,2,0,0
27:,2,3,5,0,2,0
27:,2,4,5,4,0,2
28:,1,2,14,0,0,0
28:,2,2,7,0,0,0
28:,2,3,5,2,0,0
28:,2,4,4,0,2,0
28:,2,4,5,4,2,0
30:,2,3,5,0,0,0
30:,2,4,4,2,0,0
31:,2,3,6,2,2,0
31:,2,4,5,0,2,2
32:,2,4,4,0,0,0
32:,2,4,5,4,0,0
33:,2,3,6,0,2,0
34:,2,3,6,2,0,0
34:,2,4,5,2,2,0
34:,2,4,6,4,0,2
35:,2,4,5,0,0,2
36:,2,3,6,0,0,0
36:,2,4,5,0,2,0
36:,2,4,6,4,2,0
36:,2,5,5,4,2,2
37:,2,5,5,4,2,0
38:,2,4,5,2,0,0
38:,2,4,6,0,2,2
39:,2,3,7,0,2,0
40:,2,3,7,2,0,0
40:,2,4,5,0,0,0
40:,2,4,6,2,0,2
40:,2,4,6,4,0,0
41:,2,4,7,4,0,2
42:,2,3,7,0,0,0
42:,2,4,6,0,0,2
42:,2,4,6,2,2,0
42:,2,5,5,4,0,0
43:,2,3,8,2,2,0
44:,2,4,6,0,2,0
45:,2,4,7,0,2,2
45:,2,5,5,0,2,0
45:,2,5,6,4,2,2
46:,2,3,8,2,0,0
46:,2,4,6,2,0,0
46:,2,5,6,4,0,2
47:,2,4,7,2,0,2
47:,2,5,6,4,2,0
48:,2,3,8,0,0,0
48:,2,4,6,0,0,0
48:,2,5,5,2,0,0
49:,2,3,9,2,2,0
49:,2,4,7,0,0,2
49:,2,5,6,0,2,2
50:,2,4,7,2,2,0
51:,2,3,9,0,2,0
52:,2,3,9,2,0,0
52:,2,5,6,2,0,2
52:,2,5,6,4,0,0
53:,2,5,6,2,2,0
54:,2,3,9,0,0,0
54:,2,4,7,2,0,0
54:,2,5,6,0,0,2
54:,2,5,7,4,2,2
55:,2,3,10,2,2,0
55:,2,5,6,0,2,0
55:,2,5,7,4,0,2
56:,2,4,7,0,0,0
56:,2,4,8,4,0,0
57:,2,3,10,0,2,0
58:,2,3,10,2,0,0
58:,2,4,8,2,2,0
58:,2,5,6,2,0,0
58:,2,5,7,0,2,2
60:,2,3,10,0,0,0
60:,2,4,9,4,2,0
60:,2,5,6,0,0,0
61:,2,5,7,2,0,2
62:,2,4,8,2,0,0
62:,2,5,7,4,0,0
63:,2,5,7,0,0,2
63:,2,5,7,2,2,0
64:,2,4,8,0,0,0
66:,2,4,9,2,2,0
68:,2,4,9,0,2,0
68:,2,4,10,4,2,0
68:,2,5,7,2,0,0
70:,2,4,9,2,0,0
70:,2,5,7,0,0,0
72:,2,4,9,0,0,0
72:,2,4,10,4,0,0
72:,2,5,8,4,0,0
74:,2,4,10,2,2,0
76:,2,4,10,0,2,0
77:,2,5,9,4,2,0
78:,2,4,10,2,0,0
78:,2,5,8,2,0,0
80:,2,4,10,0,0,0
80:,2,4,11,4,0,0
80:,2,5,8,0,0,0
82:,2,4,11,2,2,0
82:,2,5,9,4,0,0
85:,2,5,9,0,2,0
86:,2,4,11,2,0,0
87:,2,5,10,4,2,0
88:,2,4,11,0,0,0
88:,2,4,12,4,0,0
88:,2,5,9,2,0,0
90:,2,4,12,2,2,0
90:,2,5,9,0,0,0
92:,2,4,13,4,2,0
92:,2,5,10,4,0,0
93:,2,5,10,2,2,0
94:,2,4,12,2,0,0
95:,2,5,10,0,2,0
96:,2,4,12,0,0,0
96:,2,4,13,4,0,0
98:,2,4,13,2,2,0
98:,2,5,10,2,0,0
100:,2,4,13,0,2,0
100:,2,4,14,4,2,0
100:,2,5,10,0,0,0
102:,2,4,13,2,0,0
104:,2,4,13,0,0,0
104:,2,4,14,4,0,0
106:,2,4,14,2,2,0
108:,2,4,14,0,2,0
110:,2,4,14,2,0,0
112:,2,4,14,0,0,0
\end{verbatim}

}


\end{document}
