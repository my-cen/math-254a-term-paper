\documentclass[10pt, letterpaper]{article}
\usepackage{amsmath}
\usepackage{array}
\usepackage[margin=1in]{geometry}

\title{Appendix A: Tables of Quadratic Forms}
\date{}
\author{}

\begin{document}
    
\maketitle

\begin{table}[ht]
    \centering
    \begin{tabular}{|c|c|c|c|c|c|c|}
    \hline
     & 3D Escalator & Truant & Nos. not represented\footnotemark & If $m$ & Subtract & Check up to \\ \hline
    (1) & $\begin{bmatrix} 1 & 0 & 0 \\ 0 & 1 & 0 \\ 0 & 0 & 1 \end{bmatrix}$ & 7 & $2^e u_7$ & $\not\equiv 0 \pmod{8}$ & $m$ or $4m$ & 112 \\
     & & & & $\equiv 0 \pmod{8}$ & does not arise & - \\ \hline
    (2) & $\begin{bmatrix} 1 & 0 & 0 \\ 0 & 1 & 0 \\ 0 & 0 & 2 \end{bmatrix}$ & 14 & $2^d u_7$ & $\not\equiv 0 \pmod{16}$ & $m$ or $4m$ & 224 \\
     & & & & $\equiv 0 \pmod{16}$ & does not arise & - \\ \hline
    (3) & $\begin{bmatrix} 1 & 0 & 0 \\ 0 & 1 & 0 \\ 0 & 0 & 3 \end{bmatrix}$ & 6 & $3^d u_-$ & $\not\equiv 0 \pmod{9}$ & $m, 4m,$ or $16m$ & 864 \\
     & & & & $\equiv 0 \pmod{9}$ & does not arise & - \\ \hline
    (4) & $\begin{bmatrix} 1 & 0 & 0 \\ 0 & 2 & 0 \\ 0 & 0 & 2 \end{bmatrix}$ & 7 & $2^e u_7$ & $\not\equiv 0 \pmod{8}$ & $m$ or $4m$ & 112 \\
     & & & & $\equiv 0 \pmod{8}$ & [See Table 2] & - \\ \hline
    (5) & $\begin{bmatrix} 1 & 0 & 0 \\ 0 & 2 & 0 \\ 0 & 0 & 3 \end{bmatrix}$ & 10 & $2^d u_5$ & $\not\equiv 0 \pmod{16}$ & $m$ or $4m$ & 1440 \\
     & & & & $\equiv 0 \pmod{16}$ & does not arise & - \\ \hline
    (6) & $\begin{bmatrix} 1 & 0 & 0 \\ 0 & 2 & 1 \\ 0 & 1 & 4 \end{bmatrix}$ & 7 & $7^d u_-$ & $\not\equiv 0, 3, 9 \pmod{12}$ & $m, 4m,$ or $9m$ & 3087 \\
    & & & & $\& \not\equiv 0 \pmod{49}$ & & \\
     & & & or $7, 10 \pmod{12}$ & $\equiv 0 \pmod{49}$ & does not arise & - \\
     & & & & $\equiv 0, 3, 9 \pmod{12}$ & [See Table 2] & - \\ \hline
    (7) & $\begin{bmatrix} 1 & 0 & 0 \\ 0 & 2 & 0 \\ 0 & 0 & 4 \end{bmatrix}$ & 14 & $2^d u_7$ & $\not\equiv 0 \pmod{16}$ & $m$ or $4m$ & 224 \\
     & & & & $\equiv 0 \pmod{8}$ & [See Table 2] & - \\ \hline
    (8) & $\begin{bmatrix} 1 & 0 & 0 \\ 0 & 2 & 1 \\ 0 & 1 & 5 \end{bmatrix}$ & 7 & $2^e u_7$ & $\not\equiv 0 \pmod{8}$ & $m$ or $4m$ & 252 \\
     & & & & $\equiv 0 \pmod{8}$ & does not arise & - \\ \hline
    (9) & $\begin{bmatrix} 1 & 0 & 0 \\ 0 & 2 & 0 \\ 0 & 0 & 5 \end{bmatrix}$ & 10 & $5^d u_-$ & $\not\equiv 0 \pmod{25}$ & $m$ or $4m$ & 4000 \\
     & & & & $\equiv 0 \pmod{25}$ & does not arise & - \\ \hline
    \end{tabular}
    \caption{Proving universality of 4D escalators in nonexceptional cases}
    \label{tab:my_table}
\end{table}
\footnotetext{$p^d$ denotes an odd power of $p$, $p^e$ denotes an even power of $p$, $u_1,u_3,u_5,u_7$ are $1,3,5,7$ modulo $8$, $u_+,u_-$ are quadratic residues/non-residues modulo $p$}

\begin{table}[ht]
    \centering
    \begin{tabular}{|c|c|p{1in}|c|c|c|}
    \hline
    Exceptional Lattice & New unique-in-genus lattice & Nos. not repr. & $m$ & Subtract & Check up to \\
    \hline
    $\begin{bmatrix} 1 & 0 & 0 & 2 \\ 0 & 2 & 0 & 1 \\ 0 & 0 & 2 & 1 \\ 2 & 1 & 1 & 7 \end{bmatrix}$ & $\begin{bmatrix} 1 & 0 & 0 \\ 0 & 2 & 1 \\ 0 & 1 & 3 \end{bmatrix}$ & $5^d u_+$ & 40 & $m$ or $4m$ & 160 \\
    \hline
    $\begin{bmatrix} 1 & 0 & 0 & 0 \\ 0 & 2 & 0 & 1 \\ 0 & 0 & 2 & 1 \\ 0 & 1 & 1 & 7 \end{bmatrix}$ & $\begin{bmatrix} 2 & 0 & 1 \\ 0 & 2 & 1 \\ 1 & 1 & 7 \end{bmatrix}$ & $2^e u_1, 2^e u_5,$ \newline $2^d u_3, 2^d u_7, 3^d u_+$ & 1 & $m$ & 14 \\ \hline
    $\begin{bmatrix} 1 & 0 & 0 & 1 \\ 0 & 2 & 1 & 0 \\ 0 & 1 & 4 & 3 \\ 1 & 0 & 3 & 7 \end{bmatrix}$ & $\begin{bmatrix} 2 & 1 & 1 \\ 1 & 4 & 0 \\ 1 & 0 & 4 \end{bmatrix}$ & $2^d u_7$ & 1 & $m, 4m, \text{or } 9m$ & 9 \\
    \hline
    $\begin{bmatrix} 1 & 0 & 0 & 0 \\ 0 & 2 & 1 & 1 \\ 0 & 1 & 4 & 0 \\ 0 & 1 & 0 & 7 \end{bmatrix}$ \footnotemark & $\begin{bmatrix} 2 & 0 & 0 \\ 0 & 4 & 2 \\ 0 & 2 & 10 \end{bmatrix}$ & $2^d u_7$ & 90 & $m \text{ or } 4m$ & 504 \\
    \hline
    $\begin{bmatrix} 1 & 0 & 0 & 1 \\ 0 & 2 & 0 & 0 \\ 0 & 0 & 4 & 2 \\ 1 & 0 & 2 & 14 \end{bmatrix}$ & $\begin{bmatrix} 1 & 0 & 0 \\ 0 & 4 & 2 \\ 0 & 2 & 13 \end{bmatrix}$ & $2^d u_5, 2^e u_3$ & 2 & $m \text{ or } 4m$ & 8 \\
    \hline
    \end{tabular}
    \caption{Proving universality of 4D escalators in exceptional cases}
\end{table}
\footnotetext{This only shows even numbers are represented, but the original argument in Table 1 shows odd numbers are always represented.}

\begin{verbatim}
Table 3: The 207 four-dimensional escalators
where D:,a,b,c,d,e,f denotes the quadratic form
[ 1  0   0   0  ]
[ 0  a  f/2 e/2 ]
[ 0 f/2  b  d/2 ]
[ 0 e/2 d/2  c  ]
where
[  a  f/2 e/2 ]
[ f/2  b  d/2 ]
[ e/2 d/2  c  ]
has determinant D:
--------------------
1:,1,1,1,0,0,0
2:,1,1,2,0,0,0
3:,1,1,3,0,0,0
3:,1,2,2,2,0,0
4:,1,1,4,0,0,0
4:,1,2,2,0,0,0
4:,2,2,2,2,2,0
5:,1,1,5,0,0,0
5:,1,2,3,2,0,0
6:,1,1,6,0,0,0
6:,1,2,3,0,0,0
6:,2,2,2,2,0,0
7:,1,1,7,0,0,0
7:,1,2,4,2,0,0
7:,2,2,3,2,0,2
8:,1,2,4,0,0,0
8:,1,3,3,2,0,0
8:,2,2,2,0,0,0
8:,2,2,3,2,2,0
9:,1,2,5,2,0,0
9:,1,3,3,0,0,0
9:,2,2,3,0,0,2
10:,1,2,5,0,0,0
10:,2,2,3,2,0,0
10:,2,2,4,2,0,2
11:,1,2,6,2,0,0
11:,1,3,4,2,0,0
12:,1,2,6,0,0,0
12:,1,3,4,0,0,0
12:,2,2,3,0,0,0
12:,2,2,4,0,0,2
13:,2,2,5,2,0,2
13:,2,3,3,2,2,0
14:,1,2,7,0,0,0
14:,1,3,5,2,0,0
14:,2,2,4,2,0,0
15:,1,2,8,2,0,0
15:,1,3,5,0,0,0
15:,2,2,5,0,0,2
15:,2,3,3,0,2,0
16:,1,2,8,0,0,0
16:,2,2,4,0,0,0
16:,2,3,3,2,0,0
17:,1,2,9,2,0,0
17:,1,3,6,2,0,0
17:,2,3,4,0,2,2
18:,1,2,9,0,0,0
18:,1,3,6,0,0,0
18:,2,2,5,2,0,0
18:,2,3,3,0,0,0
18:,2,3,4,2,0,2
19:,1,2,10,2,0,0
19:,2,3,4,2,2,0
20:,1,2,10,0,0,0
20:,2,2,5,0,0,0
20:,2,2,6,2,2,0
20:,2,4,4,4,2,0
21:,2,3,4,0,2,0
22:,1,2,11,0,0,0
22:,2,2,6,2,0,0
22:,2,3,4,2,0,0
22:,2,3,5,0,2,2
23:,1,2,12,2,0,0
23:,2,3,5,2,0,2
24:,1,2,12,0,0,0
24:,2,2,6,0,0,0
24:,2,2,7,2,2,0
24:,2,3,4,0,0,0
24:,2,4,4,0,2,2
24:,2,4,4,4,0,0
25:,1,2,13,2,0,0
25:,2,3,5,2,2,0
26:,1,2,13,0,0,0
26:,2,2,7,2,0,0
26:,2,4,4,2,2,0
27:,1,2,14,2,0,0
27:,2,3,5,0,2,0
27:,2,4,5,4,0,2
28:,1,2,14,0,0,0
28:,2,2,7,0,0,0
28:,2,3,5,2,0,0
28:,2,4,4,0,2,0
28:,2,4,5,4,2,0
30:,2,3,5,0,0,0
30:,2,4,4,2,0,0
31:,2,3,6,2,2,0
31:,2,4,5,0,2,2
32:,2,4,4,0,0,0
32:,2,4,5,4,0,0
33:,2,3,6,0,2,0
33:,2,4,5,2,0,2
34:,2,3,6,2,0,0
34:,2,4,5,2,2,0
34:,2,4,6,4,0,2
35:,2,4,5,0,0,2
36:,2,3,6,0,0,0
36:,2,4,5,0,2,0
36:,2,4,6,4,2,0
36:,2,5,5,4,2,2
37:,2,5,5,4,2,0
38:,2,4,5,2,0,0
38:,2,4,6,0,2,2
39:,2,3,7,0,2,0
40:,2,3,7,2,0,0
40:,2,4,5,0,0,0
40:,2,4,6,2,0,2
40:,2,4,6,4,0,0
41:,2,4,7,4,0,2
42:,2,3,7,0,0,0
42:,2,4,6,0,0,2
42:,2,4,6,2,2,0
42:,2,5,5,4,0,0
43:,2,3,8,2,2,0
43:,2,5,5,2,0,2
44:,2,4,6,0,2,0
45:,2,4,7,0,2,2
45:,2,5,5,0,2,0
45:,2,5,6,4,2,2
46:,2,3,8,2,0,0
46:,2,4,6,2,0,0
46:,2,5,6,4,0,2
47:,2,4,7,2,0,2
47:,2,5,6,4,2,0
48:,2,3,8,0,0,0
48:,2,4,6,0,0,0
48:,2,5,5,2,0,0
49:,2,3,9,2,2,0
49:,2,4,7,0,0,2
49:,2,5,6,0,2,2
50:,2,4,7,2,2,0
50:,2,5,5,0,0,0
51:,2,3,9,0,2,0
52:,2,3,9,2,0,0
52:,2,5,6,2,0,2
52:,2,5,6,4,0,0
53:,2,5,6,2,2,0
54:,2,3,9,0,0,0
54:,2,4,7,2,0,0
54:,2,5,6,0,0,2
54:,2,5,7,4,2,2
55:,2,3,10,2,2,0
55:,2,5,6,0,2,0
55:,2,5,7,4,0,2
56:,2,4,7,0,0,0
56:,2,4,8,4,0,0
57:,2,3,10,0,2,0
58:,2,3,10,2,0,0
58:,2,4,8,2,2,0
58:,2,5,6,2,0,0
58:,2,5,7,0,2,2
60:,2,3,10,0,0,0
60:,2,4,9,4,2,0
60:,2,5,6,0,0,0
61:,2,5,7,2,0,2
62:,2,4,8,2,0,0
62:,2,5,7,4,0,0
63:,2,5,7,0,0,2
63:,2,5,7,2,2,0
64:,2,4,8,0,0,0
66:,2,4,9,2,2,0
67:,2,5,8,4,2,0
68:,2,4,9,0,2,0
68:,2,4,10,4,2,0
68:,2,5,7,2,0,0
70:,2,4,9,2,0,0
70:,2,5,7,0,0,0
72:,2,4,9,0,0,0
72:,2,4,10,4,0,0
72:,2,5,8,4,0,0
74:,2,4,10,2,2,0
76:,2,4,10,0,2,0
77:,2,5,9,4,2,0
78:,2,4,10,2,0,0
78:,2,5,8,2,0,0
80:,2,4,10,0,0,0
80:,2,4,11,4,0,0
80:,2,5,8,0,0,0
82:,2,4,11,2,2,0
82:,2,5,9,4,0,0
83:,2,5,9,2,2,0
85:,2,5,9,0,2,0
86:,2,4,11,2,0,0
87:,2,5,10,4,2,0
88:,2,4,11,0,0,0
88:,2,4,12,4,0,0
88:,2,5,9,2,0,0
90:,2,4,12,2,2,0
90:,2,5,9,0,0,0
92:,2,4,13,4,2,0
92:,2,5,10,4,0,0
93:,2,5,10,2,2,0
94:,2,4,12,2,0,0
95:,2,5,10,0,2,0
96:,2,4,12,0,0,0
96:,2,4,13,4,0,0
98:,2,4,13,2,2,0
98:,2,5,10,2,0,0
100:,2,4,13,0,2,0
100:,2,4,14,4,2,0
100:,2,5,10,0,0,0
102:,2,4,13,2,0,0
104:,2,4,13,0,0,0
104:,2,4,14,4,0,0
106:,2,4,14,2,2,0
108:,2,4,14,0,2,0
110:,2,4,14,2,0,0
112:,2,4,14,0,0,0
\end{verbatim}
\clearpage
\begin{verbatim}
Table 4: Non-universal quaternary integer-matrix escalators
------------------------------------------------------------
[ 1 0 0 0 ]
[ 0 2 0 1 ]
[ 0 0 3 0 ]
[ 0 1 0 4 ]
Truant: 10

[ 1 0 0 0 ]
[ 0 2 1 0 ]
[ 0 1 4 1 ]
[ 0 0 1 5 ]
Truant: 10

[ 1 0 0 0 ]
[ 0 2 1 0 ]
[ 0 1 5 1 ]
[ 0 0 1 5 ]
Truant: 15

[ 1 0 0 0 ]
[ 0 2 0 0 ]
[ 0 0 5 0 ]
[ 0 0 0 5 ]
Truant: 15

[ 1 0 0 0 ]
[ 0 2 0 1 ]
[ 0 0 5 2 ]
[ 0 1 2 8 ]
Truant: 15

[ 1 0 0 0 ]
[ 0 2 0 1 ]
[ 0 0 5 1 ]
[ 0 1 1 9 ]
Truant: 15
----------
The number of nonuniversal quaternary escalators is 6.
\end{verbatim}
\clearpage
\begin{verbatim}
Table 5: Truants of three-dimensional integer-matrix escalators.
-----------------------------------------------------------------
Quadratic form in 3 variables over Integer Ring with coefficients:
[ 1 0 0 ]
[ 0 1 0 ]
[ 0 0 1 ]
Truant: 7
----------
Quadratic form in 3 variables over Integer Ring with coefficients:
[ 1 0 0 ]
[ 0 1 0 ]
[ 0 0 2 ]
Truant: 14
----------
Quadratic form in 3 variables over Integer Ring with coefficients:
[ 1 0 0 ]
[ 0 1 0 ]
[ 0 0 3 ]
Truant: 6
----------
Quadratic form in 3 variables over Integer Ring with coefficients:
[ 1 0 0 ]
[ 0 2 0 ]
[ 0 0 2 ]
Truant: 7
----------
Quadratic form in 3 variables over Integer Ring with coefficients:
[ 1 0 0 ]
[ 0 2 0 ]
[ 0 0 3 ]
Truant: 10
----------
Quadratic form in 3 variables over Integer Ring with coefficients:
[ 1 0 0 ]
[ 0 2 1 ]
[ 0 1 4 ]
Truant: 7
----------
Quadratic form in 3 variables over Integer Ring with coefficients:
[ 1 0 0 ]
[ 0 2 0 ]
[ 0 0 4 ]
Truant: 14
----------
Quadratic form in 3 variables over Integer Ring with coefficients:
[ 1 0 0 ]
[ 0 2 1 ]
[ 0 1 5 ]
Truant: 7
----------
Quadratic form in 3 variables over Integer Ring with coefficients:
[ 1 0 0 ]
[ 0 2 0 ]
[ 0 0 5 ]
Truant: 10
----------
\end{verbatim}

\end{document}
